%  ========================================================================
%  Copyright (c) 1985 The University of Washington
%
%  Licensed under the Apache License, Version 2.0 (the "License");
%  you may not use this file except in compliance with the License.
%  You may obtain a copy of the License at
%
%      http://www.apache.org/licenses/LICENSE-2.0
%
%  Unless required by applicable law or agreed to in writing, software
%  distributed under the License is distributed on an "AS IS" BASIS,
%  WITHOUT WARRANTIES OR CONDITIONS OF ANY KIND, either express or implied.
%  See the License for the specific language governing permissions and
%  limitations under the License.
%  ========================================================================
%

% Documentation for University of Washington thesis LaTeX document class
% by Jim Fox
% fox@washington.edu
%
%    Revised for version 2015/03/03 of uwthesis.cls
%    Revised, 2016/11/22, for cleanup of sample copyright and title pages
%
%    This document is contained in a single file ONLY because
%    I wanted to be able to distribute it easily.  A real thesis ought
%    to be contained on many files (e.g., one for each chapter, at least).
%
%    To help you identify the files and sections in this large file
%    I use the string '==========' to identify new files.
%
%    To help you ignore the unusual things I do with this sample document
%    I try to use the notation
%       
%    % --- sample stuff only -----
%    special stuff for my document, but you don't need it in your thesis
%    % --- end-of-sample-stuff ---


%    Printed in twoside style now that that's allowed
%
 
\documentclass [11pt, proquest] {uwthesis}[2016/11/22]
 
%
% The following line would print the thesis in a postscript font 

% \usepackage{natbib}
% \def\bibpreamble{\protect\addcontentsline{toc}{chapter}{Bibliography}}

\setcounter{tocdepth}{1}  % Print the chapter and sections to the toc
 

% ==========   Local defs and mods
%

% --- sample stuff only -----
% These format the sample code in this document

\usepackage{alltt}  % 
\newenvironment{demo}
  {\begin{alltt}\leftskip3em
     \def\\{\ttfamily\char`\\}%
     \def\{{\ttfamily\char`\{}%
     \def\}{\ttfamily\char`\}}}
  {\end{alltt}}
 
% metafont font.  If logo not available, use the second form
%
% \font\mffont=logosl10 scaled\magstep1
\let\mffont=\sf
% --- end-of-sample-stuff ---
 
\usepackage{amsmath}

\begin{document}
 
% ==========   Preliminary pages
%
% ( revised 2012 for electronic submission )
%

\prelimpages
 
%
% ----- copyright and title pages
%
\Title{Simple Models and Electron Microscopies Describing Aggregates of Metal Nanoparticles}
\Author{Nicholas P. Montoni}
\Year{2018}
\Program{Department of Chemistry}

\Chair{David Masiello}{Associate Professor}{Department of Chemistry}
\Signature{David Ginger}
\Signature{Sarah Keller}
\Signature{Elizabeth Nance}

\copyrightpage

\titlepage  

 
%
% ----- signature and quoteslip are gone
%

%
% ----- abstract
%


\setcounter{page}{-1}
\abstract{%
This sample dissertation is an aid to students who are attempting
to format their theses with \LaTeX, a sophisticated
text formatter widely used by mathematicians and scientists everywhere.
 
\begin{itemize}
\item It describes the use of a specialized
macro package developed specifically for thesis production
at the University.
The macros customize \LaTeX\ for the correct thesis style,
allowing the student to concentrate on the substance of
his or her text.%
\footnote{See Appendix A to obtain the source to this
 thesis and the class file.}
\item It demonstrates the solutions to a variety of
formatting challenges found in thesis production.
\item It serves as a template for a real dissertation.
\end{itemize}
}
 
%
% ----- contents & etc.
%
\tableofcontents
\listoffigures
%\listoftables  % I have no tables
 
%
% ----- glossary 
%
\chapter*{Glossary}      % starred form omits the `chapter x'
\addcontentsline{toc}{chapter}{Glossary}
\thispagestyle{plain}
%
\begin{glossary}
\item[argument] replacement text which customizes a \LaTeX\ macro for
each particular usage.
\item[back-up] a copy of a file to be used when catastrophe strikes
the original.  People who make no back-ups deserve
no sympathy.
\item[control sequence] the normal form of a command to \LaTeX.
\item[delimiter] something, often a character, that indicates
the beginning and ending of an argument.
More generally, a delimiter is a field separator.
\item[document class] a file of macros that tailors \LaTeX\ for
a particular document.  The macros described by this thesis
constitute a document class.
\item[document option] a macro or file of macros
that further modifies \LaTeX\ for
a particular document.  The option {\tt[chapternotes]}
constitutes a document option.
\item[figure] illustrated material, including graphs,
diagrams, drawings and photographs.
\item[font] a character set (the alphabet plus digits
and special symbols) of a particular size and style.  A couple of fonts
used in this thesis are twelve point roman and {\sl twelve point roman
slanted}.
\item[footnote] a note placed at the bottom of a page, end of a chapter,
or end of a thesis that comments on or cites a reference
for a designated part of the text.
\item[formatter] (as opposed to a word-processor) arranges printed
material according to instructions embedded in the text.
A word-processor, on the other hand, is normally controlled
by keyboard strokes that move text about on a display.
\item[\LaTeX] simply the ultimate in computerized typesetting.
\item[macro]  a complex control sequence composed of 
other control sequences.
\item[pica] an archaic unit of length.  One pica is twelve points and
six picas is about an inch.
\item[point] a unit of length.  72.27 points equals one inch.
\item[roman]  a conventional printing typestyle using serifs.
the decorations on the ends of letter strokes.
This thesis is set in roman type.
\item[rule] a straight printed line; e.g., \hrulefill.
\item[serif] the decoration at the ends of letter strokes.
\item[table] information placed in a columnar arrangement.
\item[thesis] either a master's thesis or a doctoral dissertation.
This document also refers to itself as a thesis, although it
really is not one.
 
\end{glossary}
 
%
% ----- acknowledgments
%
\acknowledgments{% \vskip2pc
  % {\narrower\noindent
  The author wishes to express sincere appreciation to
  mom and dad, research group, community community community.
  % \par}
}

%
% ----- dedication
%
\dedication{\begin{center}to my queens, Barbra Streisand and Ruth Bader Ginsburg\end{center}}

%
% end of the preliminary pages
 
 
 
%
% ==========      Text pages
%

\textpages
 
% ========== Chapter 1
 
\chapter {Introduction}
 
A brief history of plasmonics. Contextualization of my goals, like catalysis and negative-index materials. Cool applications.

When one thinks of a metal, a few images come to mind: electrical wires that power modern technology or a reflection in a mirror. These images arise from the understanding that metals are particularly good conductors; the conduction electrons in metals are generally free to move about uninhibited. This conductivity is what makes mirrors possible. The conduction electrons in a metal will rearrange themselves to screen incoming electric fields, such as those from light. Of course, in reality, no metals are perfect conductors. This means that light actually can penetrate metal up to a distance known as the metal's skin-depth, on the order of 10-100 nanometers. Now, imagine a piece of metal about that size (a nanoparticle is to the reader as the reader is to Earth); an incoming light wave can penetrate the nanoparticle entirely, perfectly polarizing its conduction electrons. For the most part, the electrons are unable to leave the nanoparticle, so they collect on the surface. When the field is removed, the electrons drift back to their equilibrium positions, overshoot, and swing to the opposite surface. This collective and coherent oscillation of the electron plasma is known as a localized surface plasmon resonance (LSPR). For most metal nanoparticles, the LSPR frequency is somewhere between the near IR and the near UV.

Though first predicted in 1952 by Bohm and Pines, LSPRs have had great impact on arts and culture since the Gothic movements of Western Europe. The vibrant colors of stained glass are the result of colloidal gold nanoparticles suspended in the glass. The color depends on the particle size and density. But why? Well, that's part of what we're going to learn in this introduction: why the size and aggregation scheme of nanoparticles affect their optical properties. To do this, we'll need to understand the optical properties of individual nanoparticles and investigate how nanoparticles interact in aggregates. Not only will this explain the phenomenon responsible for the colors of stained glass, but it will also inform the description and characterization of metal nanoparticle aggregates that make up the main text of this thesis.

\section{LSPRs as Harmonic Oscillators}

In order to understand the harmonic oscillator dynamics of a metal nanosphere, we must consider its polarizability,

\begin{equation}
\alpha(\omega) = a^3\frac{\ell(\varepsilon(\omega)-\varepsilon_b)}{\ell(\varepsilon(\omega)+\varepsilon_b)+\varepsilon_b}.
\label{polarizability_1}
\end{equation}

This is the frequency-dependent polarizability from the Clausius-Mossotti relation, defined by frequency-dependent dielectric function 

\begin{equation}
\varepsilon(\omega) = \varepsilon_{\infty} - \frac{\omega_p^2}{\omega^2+\textrm{i}\gamma\omega}
\label{dielectric}
\end{equation}

where $\omega_p^2 = 4\pi ne^2/m$ is the plasma frequency and $\gamma$ is the bulk damping rate. Plugging Equation~\ref{dielectric} into Equation~\ref{polarizability_1} and doing some algebra, we can coax the polarizability into a slightly more useful form. We will also, for now, set $\ell = 1$ because we are only considering the dipole response of the sphere, and we will also set $\varepsilon_b = 1$ because we are concerned only with spheres in vacuum.

\begin{equation}
\alpha(\omega) = a^3\left[\frac{\left(\omega^2+\textrm{i}\gamma\omega\right)\left(\frac{\varepsilon_{\infty}-1}{\varepsilon_{\infty}+2}\right)-\omega_{\textrm{sp}}^2}{\omega^2+\textrm{i}\gamma\omega-\omega_{\textrm{sp}}^2}\right]
\label{polarizability_2}
\end{equation}

Here, $\omega_{\textrm{sp}}^2 = \omega_p^2/\varepsilon_{\infty}+2$ is the surface plasmon frequency for a dipole. Now that we have this rather complicated-looking function of frequency, we want to know a bit about the behavior of the polarizability in the time domain. In order to do that, we're going to Fourier transform the polarizability. However, the denominator of the polarizability contains two poles, so we're going to have to do a contour integral using the residue theorem to actually compute the Fourier transform. Equation~\ref{polarizability_2} has poles at $\omega = -\textrm{i}\gamma/2 \pm \sqrt{\omega_\textrm{sp}^2-\gamma^2/4}$. We are trying to find

\begin{equation}
\alpha(t) = \int_{-\infty}^{\infty}\frac{d\omega}{2\pi}\alpha(\omega)e^{\textrm{i}\omega t} = 2\pi\textrm{i}\sum\textrm{Res}
\label{res_theorem}
\end{equation}

Because it has two poles, the integrand in Equation~\ref{res_theorem} has two residues

\begin{equation}
R_{\pm} = \pm\frac{1}{2\pi}\frac{\omega_{\textrm{sp}}^2\left(\frac{\varepsilon_{\infty}-1}{\varepsilon_{\infty}+2}-1\right)}{2\sqrt{\omega_\textrm{sp}^2-\gamma^2/4}}e^-{\gamma t/2}e^{\pm\textrm{i}\sqrt{\omega_\textrm{sp}^2-\gamma^2/4}t}
\label{residues}
\end{equation}

Plugging each of the residues into the residue theorem, doing some algebra, and remembering that $ 2\textrm{i}\textrm{sin}(\phi) = e^{\textrm{i}\phi} - e^{\textrm{i}\phi} $, results in the following expression for the polarizability.

\begin{equation}
\alpha(t) = a^3\left(\frac{3}{\varepsilon_{\infty}+2}\right)\frac{\omega_{\textrm{sp}}^2}{\sqrt{\omega_\textrm{sp}^2-\gamma^2/4}}e^{-\gamma t/2}sin(\sqrt{\omega_\textrm{sp}^2-\gamma^2/4}t)
\label{polar_time}
\end{equation}

Equation~\ref{polar_time} has two important pieces. The first is the sinusoidal term, oscillating at frequency $\sqrt{\omega_\textrm{sp}^2-\gamma^2/4}$. The second is the exponential term, decaying with width $\gamma/2$. These two terms show that a plasmon has time dynamics consistent with a harmonic oscillator. Now that we understand how one dipole plasmon behaves, let us move on to considering multiple.

\section{Multiple Metal Nanoparticles}

We need a whole paragraph AT LEAST on history.
Who to cite: Lukas, von Plessen, El Sayed, Schatz, others studying plasmon hybridization

Let's talk about plasmon hybridization theory. I can make a really wonderful analogy for the chemists in the room: this is basically molecular orbital theory. What do I mean by that? Well, take any two atomic orbitals. Maybe two s-orbitals on two hydrogen atoms. Bring them close to each other. What happens? They mix - two atomic orbitals become two molecular orbitals, each with different properties. The lower energy mode with significant overlap is called the bonding ($\sigma$) orbital. The higher-energy mode with no overlap is called the anti-bonding ($\sigma^*$) orbital. Pairs of LSPRs behave similarly. When brought close to each other, they hybridize and produce new normal modes, one lower energy and one higher energy. If one LSPR is a harmonic oscillator, then two or more can be treated like a system of coupled harmonic oscillators. The way that dipole LSPRs couple is by a pairwise interaction between each dipole and the field produced by each other dipole. This interaction is mediated by something called the dipole relay tensor 
\begin{equation}
\boldsymbol{\Lambda}_{ij} = \left\{\left(\frac{1}{r_{ij}^3} - \frac{ik}{r_{ij}^2}\right)\left(3\hat{\textbf{n}}_{ij}\hat{\textbf{n}}_{ij} - \textbf{1}\right) - \frac{k^2}{r_{ij}}\left(\hat{\textbf{n}}_{ij}\hat{\textbf{n}}_{ij} - \textbf{1}\right)\right\}\frac{e^{\textrm{i}kr_{ij}}}{\varepsilon_b}
\label{dipole_relay_tensor_full}
\end{equation}
where the $i$th and $j$th dipoles are displaced by $r_{ij}\hat{\textbf{n}}_{ij}$ and $k=\sqrt{\varepsilon_b}\omega/c$. To build intuition about this object, we will consider a simple example: two dipoles, separated by $s\hat{\textbf{x}}$, with equal dipole magnitudes and directions perpendicular to the direction of displacement (see Fig. blank). To further build intuition, let us also consider that $ka \ll 1$. To actually compute the interaction energy between these dipoles, we need to dot them into the dipole relay tensor.
\begin{equation}
\begin{split}
U &= -\textbf{d}_1\cdot\boldsymbol{\Lambda}_{12}\cdot\textbf{d}_2\\
&= -e^2q^2\hat{\textbf{y}}\cdot\frac{3\hat{\textbf{x}}\hat{\textbf{x}} - \textbf{1}}{s^3}\cdot\hat{\textbf{y}}\\
&= \frac{e^2q^2}{s^3}
\label{quasi_int}
\end{split}
\end{equation}
Here, $\textbf{d}=eq\hat{\textbf{y}}$. After taking the small $ka$ limit, often called the quasistatic approximation, we are left with one term in the dipole relay tensor, namely the near-field term. The interaction energy for a pair of parallel dipoles carries a positive sign, indicating that it is repulsive in nature, and depends on the magnitudes of the dipole moments and the separation distance between them. The same procedure reveals similar dependence for anti-parallel, collinear, and anti-collinear dipole orientations. In Fig. blank the quasistatic interaction energies are shown as a function of distance for each of the four dipole arrangements.

Going back to Equation \ref{dipole_relay_tensor_full} in full, we can perform the same procedure as above to compute the fully retarded interaction energy between pairs of dipoles. For the parallel arrangement, this becomes
\begin{equation}
U = e^2q^2\left(\frac{1}{s^3}-\frac{\textrm{i}k}{s^2}-\frac{k^2}{s}\right)e^{\textrm{i}ks}.
\label{int_ret}
\end{equation}
The interaction energy now has terms that depend on $k$, and through that, the oscillation frequency $\omega$. Also interesting to note is that each term carries a different sign and the entire energy carries a complex exponential. So, as a function of increasing separation distance, the individual terms in the interaction energy will change character from attractive to repulsive. To see how including retardation impacts the interaction energies between the dipole pairs, see Fig. blank. The significance of retardation effects in larger assemblies will be expanded upon in Chapter 5. In fact, we will discuss that the interaction energy depends on the collective frequency, which in turn depends on the interaction energy, requiring a self-consistent solution.

The insight gained from this exercise is that the "bonding" or "anti-bonding" character of an arrangement of dipoles actually depends on the separation distance between the dipoles and their collective frequency.

\section{List of publications}

1. Alpha-terpinene
2. Imaging hybridization
3. AuPt
4. Magnetic Hybridization 
5. Landau damping
6. Tunable magnetic plasmons
7? SEELS?

% ========== Chapter 2
 
\chapter{Stimulated Electron Energy-Loss Spectroscopy: Theory and Simulation}

Just copy in the SEELS notes.
 
The \TeX\ formatting program is the creation of
Donald Knuth of Stanford University.
It has been implemented on nearly every general purpose computer and
produces exactly the same copy on all machines.
 
\section{What is it; why is it spelled that way; 
and what do
really long section titles look like in the text and in the
Table of Contents?}
 
\TeX\ is a formatter.  A document's format is controlled
by commands embedded in the text.  
\LaTeX\ is a special version of \TeX---preloaded
with a voluminous set of macros that simplify most
formatting tasks.
 
\TeX\ uses {\it control sequences} to control
the formatting of a document.  These control sequences are usually
words or groups of letters prefaced with the backslash character
({\tt\char'134}).
For example,
Figure \ref{start-2} shows the text that printed the beginning
of this chapter.  Note the control sequence \verb"\chapter" that
instructed \TeX\ to start a new chapter, print the title, and
make an entry in the table of contents.  It is an example
of a macro defined by the \LaTeX\ macro package.
The control sequence \verb"\TeX", which prints the word \TeX,
is a standard macro from the {\it\TeX book}.
The short control sequence \verb"\\" in the title instructed \TeX\ to
break the title line at that point.
This capability is an example of an extension to \LaTeX\
provided by the uwthesis document class.
 
\begin{figure}
\begin{demo}
\uwsinglespace
\\chapter\{A Brief\\\\Description of \\TeX\}

The \\TeX\\ formatting program is the creation of
Donald Knuth of Stanford University.
\end{demo}
\label{start-2}
\caption{The beginning of the Chapter II text}
\end{figure}
 
Most of the time \TeX\ is simply building paragraphs from
text in your source files.  No control sequences are involved.
New paragraphs are indicated by a blank line in the
input file.
Hyphenation is performed automatically.
 
\section{\TeX books}
 
The primary reference for \LaTeX\ is Lamport's second edition
of the \textit{\LaTeX\ User's Guide}\cite{Lbook}.
It is easily read and should be sufficient for thesis formatting.
See also the \textsl{\LaTeX\ Companion}\cite{companion} for descriptions
of many add-on macro packages.

Although unnecessary for thesis writers, the \textsl{\TeX book}
is the primary reference for \TeX sperts worldwide.
 
\section{Mathematics}
 
The thesis class does not expand on \TeX's
or \LaTeX's
comprehensive treatment of mathematical equation printing.%
\label{c2note}\footnote{%
% a long footnote indeed.
 Although many \TeX-formatted documents contain no
 mathematics except the page numbers, it seems appropriate
 that this paper, which is in some sense about \TeX,
 ought to demonstrate an equation or two.  Here then, is a statement 
 of the {\it Nonsense Theorem}.
 
 \smallskip
 \def\RR{{\cal R\kern-.15em R}}
 {\narrower\hangindent\parindent Assume a universe $E$ and a symmetric function
  $\$$ defined on $E$, such that for each $\$^{yy}$ there exists a
  $\$^{\overline{yy}}$, where $\$^{yy} = \$^{\overline{yy}}$.
  For each element $i$ of $E$ define
  ${\cal S}(i)=\sum_i \$^{yy}+\$^{\overline{yy}}+0$.
  Then if $\RR$ is that subset of $E$ where $1+1=3$,
  for each $i$
  $$\lim_{\$\to\infty}\int {\cal S}di =
      \cases{0,&if $i\not\in\RR$;\cr
             \infty,&if $i\in\RR$.\cr}$$
  \par}} % end of the footnote
%
The {\it\TeX book}\cite{book}, {\it \LaTeX\ User's Guide}\cite{Lbook},
and {\it The \LaTeX\ Companion}\cite{companion}
thoroughly cover this topic.
 
 
\section{Languages other than English}
 
Most \LaTeX\ implementations at the University are tailored
for the English language.  However, \LaTeX\ will format many
other languages.  Unfortunately, this author has never been successful in 
learning more than a smattering of anything other than English.
Consult your department or the Tex Users Group.
\smallskip
\begin{center}
{\tt http://tug.org/},
\end{center}
\smallskip
for assistance with non-English formatting.

Unusual characters can be defined via the
font maker \hbox{\mffont METAFONT} (documented by Knuth\cite{Metafont}).
The definitions are not trivial.
Students who attempt to print a thesis with
custom fonts may soon proclaim,
 
% Note.  This is not the ideal way to print Greek
\medskip
\begin{center}
``$\mathaccent"7027\alpha\pi o\kern1pt\theta\alpha\nu\epsilon\hat\iota\nu$
\ $\theta\acute\epsilon\lambda\omega$.''
 
\end{center}
 
% ========== Chapter 3
 
\chapter{Paper 1: Prisms}
 
{\bf Abstract}

Driven by the desire to understand energy transfer between plasmonic and catalytic metals for applications such as plasmon-mediated catalysis, we examine the spatially resolved electron energy-loss spectra (EELS) of both pure Au nanoprisms and Pt-decorated Au nanoprisms. The EEL spectra and the resulting surface-plasmon mode maps reveal detailed near-field information on the coupling and energy transfer in these systems, thereby elucidating the underlying mechanism of plasmon-driven chemical catalysis in mixed-metal nanostructures. Through a combination of experiment and theory we demonstrate that although the location of the Pt decoration greatly influences the plasmons of the nanoprism, simple spatial proximity is not enough to induce significant energy transfer from the Au to the Pt. What matters more is the spectral overlap between the intrinsic plasmon resonances of the Au nanoprism and Pt decoration, which can be tuned by changing the composition or morphology of either component.

{\bf up to first two figures}

Localized surface plasmon resonances (LSPRs), the quantized oscillation of the free electron gas in metal nanoparticles, underlie a variety of applications ranging from surface-enhanced spectroscopy(1-5) and sensing(6, 7) to solar energy harvesting.(8-11) Plasmons in noble metals commonly occur in the visible part of the spectrum and can focus light to subdiffraction limited spots, thereby converting light energy from the far-field into the near-field. The exceptionally large polarizability of nanoparticles at the resonance frequency of the LSPR results in absorption cross sections that can be more than an order of magnitude larger than the particle’s physical size. There is great interest, therefore, in utilizing this light-harvesting property to drive chemical reactions(12-15) or improve solar device efficiency.(8, 10, 16) There is also a growing body of evidence indicating that LSPR excitation can drive otherwise unfavorable reactions such as the conversion of 4-aminothiophenol (4ATP) to 4,4′-dimercaptoazobenzene (DMAB),(12, 17, 18) H2 dissociation on Au,(15) liquid water splitting,(19-21) hydrocarbon conversion,(22) and gas-phase oxidation.(23)

Bimetallic systems, composed of catalytic and plasmonic metals, are especially interesting because they provide a potential route to further increase the efficiency of plasmon-driven chemical reactions. However, the plasmon resonances of common catalytic metals, e.g., Pd, Pt, and Rh, occur at energies much higher than those of noble metals and are lossy.(24) Nevertheless, recent studies demonstrate plasmon-enhanced energy transfer is applicable in catalytic chemistry.(25-27) Wang et al.,(28) for example, showed that Pd-decorated Au nanorods can catalyze a Suzuki coupling of bromobenzene and m-tolylboronic acid upon plasmon excitation. Zheng et al.(13) observed a similar phenomenon where Au nanorods and nanospheres with pendant Pt nanoparticles could catalyze H2 evolution, and they speculated that hot-electron generation via decay of the Au LSPR drives the reaction.
Electron energy-loss spectroscopy (EELS), performed in a monochromated scanning transmission electron microscopy (STEM) instrument, is especially promising for studying energy transfer between plasmonic and catalytic metals because it combines subnanometer spatial resolution with spectral resolution of approximately 100 meV.(29) Recently, for example, Li et al.(30) illustrated how EELS can spatially map energy transfer from individual plasmonic nanocubes to their semiconductor substrates; Wu et al.(31) explored the plasmonic properties of size-tunable alloy systems; and Ringe et al.(32) studied Pd-coated Au nano-octopods revealing strong EEL signals at the Pd-coated tips.

While previous work has established the viability of plasmon-assisted chemistry in mono- and bimetallic nanostructures, fundamental studies of plasmon hybridization in well-defined, mixed-metal systems are needed to elucidate the mechanisms underlying these observations. Because of their well-characterized plasmonic properties, in this Letter we investigate bare Au nanoprisms and Au nanoprisms decorated with spherelike, dendritic Pt nanoparticles (Au+Pt) in order to gain an understanding of the coupling between optical and catalytic properties in bimetallic nanostructures.(33, 34) STEM/EELS measurements and full-wave numerical EELS simulations are performed on pure Au nanoprisms and Pt-decorated Au nanoprisms, respectively, to understand how unperturbed Au plasmon modes are deformed by the location of the Pt particle. This work provides a nanoscopic view of how the plasmon mode structure of Au nanoprisms changes both spatially and spectrally in the presence of Pt and provides insight into energy transfer between Au and Pt constituents within the mixed-metal system. Using EELS simulations only, we further find that the LSPRs of an Al nanoprism will spectrally overlap with the LSPR of Pt, resulting in coupling stronger than that of the Au+Pt system.

Figures 1 and 2 display the high-angle annular dark-field (HAADF) images and compare the experimentally acquired EEL maps with EELS simulations for two different Au+Pt geometries: a 209 nm edge length Au nanoprism with a 40 nm diameter Pt particle deposited at the tip (Figure 1d) and a 198 nm edge length Au nanoprism with a 40 nm diameter Pt particle deposited at the center of the prism (Figure 2d). In each case, we compare the data obtained from the decorated particles with data from bare Au nanoprisms of the same size (Figures 1a and 2a) to study the influence of the Pt particle on the plasmon mode structure. The loss energies for the mode maps are selected using the peak maxima of the point EEL spectra when the electron beam is positioned at the tip, edge, corner, and face and on the Pt decoration. These beam positions are selected as representative points of the LSPR modes of the nanoprism and Pt decoration. All experimental nanostructures are supported on a 30 nm thick Si3N4 membrane, and all simulations are performed in vacuum.

{\bf up to fig 3}

Figures 1b and 2b show that the plasmon modes of the bare prisms evolve from dipoles localized to the corners, through those on the edges, and finally into multipolar modes as a function of increasing energy, which is in excellent agreement with previous EELS studies of plasmonic nanoprisms.(35, 36) Interestingly, a comparison of the mode maps resulting from decorated and undecorated prisms (Figure 1b,e) shows minimal changes to the mode structure upon addition of the Pt particle. The most notable difference is seen in the splitting of the dipolar mode of the bare Au nanoprism, where the presence of the Pt particle breaks the degeneracy of the dipole mode (vide infra) (Figure 1b,e). Figure 2b,e illustrates that the structure of the higher-energy modes is not strongly dependent on the Pt particle location, as the EEL maps in Figures 1b,e and 2b,e are similar. Lastly, we are able to experimentally map the LSPR of the Pt decoration at a higher energy than the modes of the nanoprism (Figures 1e and 2e).

Boundary element method calculations(37) (Figures 1c,f and 2c,f) capture the complex mode structure of these mixed-metal systems and agree well with the experimental measurements (Figures 1be and 2b,e). While there are quantitative differences in specific plasmon resonance energies between simulation and experiment due to substrate effects, we can leverage the qualitative agreement in the equivalent mode identity to explore the perturbative effects of Pt on the well-understood LSPR modes(38) of Au nanoprisms with plasmon hybridization theory. Previous work(39, 40) has shown that a basis set composed of corner-localized dipoles is sufficient to interpret the low-energy plasmon mode structure of the Au+Pt system. Therefore, we create a model composed of dipoles located on the corners of a triangular prism.(41, 42) Each corner is represented by a disk with two in-plane, orthogonal, dipole plasmons. These single-disk dipoles can be rigorously mapped onto a set of mechanical oscillators and mutually hybridized by diagonalizing the Hamiltonian(43)

\begin{equation}
H = \frac{1}{2}\sum_{i}\hbar\omega_{\textrm{sp}^i}[\textbf{P}_i^2 + \textbf{Q}_i^2] - \frac{1}{2}\sum_{i \neq j}\hbar(\omega_{\textrm{sp}}^i\omega_{\textrm{sp}}^j)^{1/2}g_{ij}(r_{ij})[3(\textbf{Q}_i\cdot\hat{\textbf{n}}_{ij}\hat{\textbf{n}}_{ij}\cdot\textbf{Q}_j)-\textbf{Q}_i\cdot\textbf{Q}_j],
\end{equation}

with distance-dependent coupling constants $g_{ij}(r_{ij}) = 3/[r_{ij}^3(\varepsilon_{\infty} + 2)]$ where $i$ and $j$ ($i, j$ = 1–6) indicate the identity of each plasmon (location and orientation); $\textbf{Q}_i$ is the coordinate of the $i$th oscillator with conjugate momentum $\textbf{P}_i$; $\omega_{\textrm{sp}}^i$ is its surface plasmon resonance frequency; $r_{ij}$ are the dimensionless distances between the $i$th and $j$th disks relative to their diameters; and $\hat{\textbf{n}}_{ij}$ is the corresponding unit vector connecting them.
In analogy to the mixing of carbon p-orbitals in benzene, the mixing of these six single-disk modes result in six hybridized modes: the lowest-energy mode having no nodes; the second lowest being two degenerate, single-node modes (dipole modes); the third lowest being two degenerate, double-node modes; and the highest-energy mode having three nodes. To produce a quantitative fit to the experimental and simulated data, we choose the resonance frequencies of the disk plasmons so that the resulting dipole plasmon resonance of the Au prism produces the experimentally measured resonance energy (∼1.00 eV).
With this simple model in hand, we are now able to explore the perturbing effects of a Pt particle. Simulation using experimentally derived dielectric data(24) predicts the plasmon resonance of the Pt particle to be near 5.00 eV, which is at a significantly higher energy than the LSPR modes of the Au nanoprism. We introduce the Pt particle to the system as a disk with two degenerate, in-plane, orthogonal dipoles with resonance energies of 5.00 eV, as dictated by experiment on single Pt particles. The plasmon resonances of the collective system are explored with the same hybridization method as before.
Figure 3 displays the hybridization model describing the interaction between the Au nanoprism dipole modes and the Pt dipole modes together with corresponding experimental EEL maps that illustrate the effect of the Pt on the Au nanoprism. Depositing the Pt particle at the center of the nanoprism gives the system 3-fold symmetry and, in turn, does not significantly perturb the LSPR modes of the nanoprism. The mode mixing does cause a small, equal net lowering of the prism-localized collective dipole modes and a small, equal net raising of the Pt-localized collective dipole modes. As the Pt particle is moved toward a corner, however, the system’s dipole modes begin to split. When the prism and the Pt-localized dipole modes are oriented in the same direction, they couple more strongly than when the dipoles are oriented antiparallel; therefore, the former is lower in energy, as expected.(41) The model further shows that this splitting is maximal when the Pt particle is as close to the corner as possible, which is consistent with experimental results.

{\bf up to fig 4}

Figure 4a compares the simulated EEL spectra of a 209 nm pure Au nanoprism (black), a 209 nm Au+Pt nanoprism (red) where the 40 nm diameter Pt decoration is at the tip of the nanoprism, and an aloof EEL spectrum of a 40 nm diameter Pt sphere (purple). The Au nanoprism is plasmonically active at energies between 0.90 and 2.40 eV, whereas the Pt sphere is active at a much higher energy (∼5.00 eV). Because of the spectral mismatch in plasmon resonance energies of Au and Pt, these metals are not expected to significantly hybridize and resonance energy transfer is unfavorable. The Pt decoration, therefore, simply acts as part of the dielectric environment, and the plasmons of the prism couple to their images in the Pt particle and vice versa. This image effect is identical for all of the Au and Pt dipole plasmons in the center-decorated system. However, in the tip-decorated system, the nanoprism dipole that is aligned with the Pt dipole couples more strongly to its image than the other orthogonal dipole and is shifted to lower energy.

{\bf up to fig. 5 and 6}

The framework presented here, while showing that the Au and Pt plasmons remain mostly uncoupled, does offer insight into the construction of systems with highly coupled plasmon modes for plasmon-mediated catalysis. Consequently, we study Al prisms(44, 45) decorated with Pt because the Al plasmons lie at energies higher than those of Au.(46) Figure 4b compares simulated EEL spectra for a 209 nm Al nanoprism (blue), a 209 nm Al+Pt nanoprism (green) with a 40 nm Pt tip decoration, and a 40 nm diameter Pt sphere (purple). The spectrum of the 3.20 eV peak in the Al+Pt system is broadened by 0.31 eV, determined from the full width at half-maximum (fwhm), with respect to the higher-energy Al prism-localized LSPR modes (∼3.20 eV), suggesting stronger hybridization with the Pt dipole plasmons than in the Au+Pt system. Both Pt and Al have low-energy interband transitions (0.90 and 1.40 eV, respectively) that are not expected to hybridize because of their poor spectral overlap.(47)
To further explore the properties of Al coupled with Pt, Figures 5 and 6 compare the magnitude of the electron-beam-induced electric near-fields of the Al+Pt and Au+Pt systems at three loss-energies: 1.20, 1.75, and 4.75 eV, as indicated in Figure 4 (dotted, vertical lines). Qualitatively similar electric near-fields would arise in response to linearly polarized plane-wave excitation because of the common polarization it shares with the field of the electron at the indicated beam position (x). The field maps reveal that at the energy of the prism-localized dipole mode (Figure 5a,b) and the energy of the Pt-localized dipole mode (Figure 5c,d), the fields around the Pt particles are similar in magnitude. This is reinforced in Figure 5e,f, which present the electric field magnitudes computed along lines (white) bisecting the Pt particle and clearly show electric fields of similar magnitudes for both systems.

{\bf up to methods}

Turning to the field magnitude in the junction at the energy of the prism-localized dipoles (Figure 6a,b), we see that the field is much larger in the Au+Pt junction than in the Al+Pt junction. This means that the junction field in the Au+Pt systems is dominated by the response of the Au nanoprism. Conversely, at the energy of the Pt-localized dipole (Figure 6c,d), the junction field in the Al+Pt system is much larger than that of Au+Pt. These observations are again reinforced by the electric field magnitudes presented in Figure 6e,f, which are computed along lines (white) running through the junction. Taken together, Figures 4–6 indicate a relationship between EEL probability and the strength of the near electric fields surrounding the mixed-metal systems observed in this Letter. These junction fields show that the high-energy modes of the nanoprism are more effectively coupled to the Pt dipole plasmon in Al+Pt than in Au+Pt; therefore, a greater capacity for plasmon-mediated chemical catalysis is predicted in the Al+Pt system. Utilizing the electric near-field simulations and data obtained through experiment and calculations, we show that these plasmonic-catalytic metal systems should drive plasmon-assisted reactions.
This work shows through electrodynamics simulations and experimental EELS the extent to which Au nanoprisms decorated with Pt deposits can transfer energy. EELS demonstrates that Pt changes the mode structure of the Au nanoprism, especially depending on the location of the Pt. Even though we show weak coupling between Au and Pt, we determine stronger coupling can be observed in systems where the LSPRs spectrally overlap. In a system such as Al+Pt, the LSPRs of Al and Pt exhibit more spectral overlap because both metals support LSPRs in the ultraviolet regime.

{\bf methods}

In a typical synthesis, fully described in the Supporting Information, reduction of H2PtCl6 on the surface of AUT (11-amino-1-undecanethiol hydrochloride)-conjugated nanoprisms resulted in the growth of approximately 1–3 Pt nanoparticles (average diameter of 30 ± 5 nm) per Au nanoprism substrate. Using high-resolution transmission electron microscopy (HRTEM), it was found that the Pt nanoparticles had a dendritic morphology (SI Figure 3). Interestingly, nucleation of the Pt nanoparticles occurred predominantly on the edges and vertices of the Au nanoprisms, likely due to the higher concentration of defects in the self-assembled monolayer at these sites.(48)
Post-synthesis, Au nanoprisms and Pt-decorated Au nanoprisms were sonicated for 5 min and 2.5 μL of each NP solution was drop-cast onto two separate (S)TEM compatible Si3N4 substrates with 30 nm thick membrane acquired from SPI Supplies. These samples were covered to eliminate contamination and left to air-dry for 3 h prior to EELS acquisition. EEL spectra were acquired in a monochromated Carl Zeiss Libra 200MC (S)TEM operated with an accelerating voltage of 200 kV. Each spectrum acquisition was executed with a collection semiangle of 12 mrad, a convergence semiangle of 9 mrad, and a dispersion of 29 meV per channel. Energy resolution, defined as the full width at half-maximum of the zero-loss peak, for each acquisition is 150 meV with the electron beam probing only the Si3N4 membrane. For each nanoprism, EEL spectrum images responsible for producing LSPR mode maps were collected by defining a region of interest (ROI) around the particle with dimensions 34 × 29 pixels, where 1 pixel is ∼3.3 × 3.3 nm2.
Experimentally obtained EELS mode maps are analyzed using Gatan Digital Micrograph software. Experimental EEL LSPR mode maps (Figures 1b,e and 2b,e) are generated by removing the background using the reflected-tail model and normalized to the zero-loss peak. LSPR mode maps were prepared by plotting spectral intensities from energy slices selected from peak maxima of the single-point spectra from the top corner, edge, bottom corner, face, and center of the Pt decoration to fully represent the LSPR modes of the bare Au nanoprism and the Pt-decorated Au nanoprism system.
For simulations, we applied the Metal Nanoparticle Boundary Element Method (MNPBEM) software.(37) MNPBEM represents nanoparticles as surfaces and discretizes each particle into a chosen number of surface elements. Maxwell’s equations are solved at each of these so-called boundary elements in order to calculate the nanoparticle’s optical properties. EELS simulations were performed on Au and Al nanoprisms with edge lengths of 195 and 209 nm and Au+Pt and Al+Pt nanoprism systems with prism edge lengths of 198 and 209 nm and decorated with Pt spheres of diameter 40 nm at the center and tip, respectively. Each system was simulated with no substrate. These simulations used tabulated dielectric data from Johnson and Christy(49) and Rakic et al.(50) Spectra were acquired at beam positions located at all three corners and edges of each system. Additionally, EEL maps were generated at each of the relevant energy windows. Finally, MNPBEM was used to generate maps of the electric field magnitude around the systems using the electron beam as a source.

{\bf remember SI and refs.}

% ========== Chapter 4
 
\chapter{Paper 2: Magnetic Plasmons}
 
 
From a given source \TeX\ will produce exactly the same document
on all computers and, if needed, on all printers.  {\it Exactly the same}
means that the various spacings, line and page breaks, and
even hyphenations will occur at the same places.

How you edit your text files and run \LaTeX\ varies
from system to system and depends on your personal preference.

\section{Running}

The author is woefully out of his depth where 
\TeX\ on Windows is concerned.  Google would be his resource.
On a linux system he types

\begin{demo}
\$\ pdflatex uwthesis
\end{demo}

and it generally works.

 
\section{Printing}
 
All implementations of \TeX\ provide the option of {\bf pdf} output,
which is all the Graduate School requires.  Even if you intend to
print a copy of your thesis create a 
{\tt pdf}.  It will print most anywhere.

\printendnotes

%
% ==========   Bibliography
%
\nocite{*}   % include everything in the uwthesis.bib file
\bibliographystyle{plain}
\bibliography{uwthesis}
%
% ==========   Appendices
%
\appendix
\raggedbottom\sloppy
 
% ========== Appendix A
 
\chapter{}
 
The uwthesis class file, {\tt uwthesis.cls}, contains the parameter settings,
macro definitions, and other \TeX nical commands which
allow \LaTeX\ to format a thesis.  
The source to
the document you are reading, {\tt uwthesis.tex},
contains many formatting examples
which you may find useful.
The bibliography database, {\tt uwthesis.bib}, contains instructions
to BibTeX to create and format the bibliography.
You can find the latest of these files on:

\begin{itemize}
\item My page.
\begin{description}
\item[] \verb%http://staff.washington.edu/fox/tex/uwthesis.html%
\end{description}

\item CTAN
\begin{description}
\item[]  \verb%http://tug.ctan.org/tex-archive/macros/latex/contrib/uwthesis/%
\item[]  (not always as up-to-date as my site)
\end{description}

\end{itemize}

\vita{Jim Fox is a Software Engineer with IT Infrastructure Division at the University of Washington.
His duties do not include maintaining this package.  That is rather
an avocation which he enjoys as time and circumstance allow.

He welcomes your comments to {\tt fox@uw.edu}.
}


\end{document}
